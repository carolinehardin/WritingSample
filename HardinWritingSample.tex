\documentclass[a4paper]{article}
\usepackage[T1]{fontenc}
\usepackage[utf8]{inputenc}
\usepackage[english]{babel}
\usepackage{bera}
\usepackage{apacite}
\bibliographystyle{apacite}

\usepackage[protrusion=true,expansion=true]{microtype} % Better typography


\makeatletter{}
\setlength{\parindent}{10ex}
\setlength{\parskip}{1em}

\renewcommand{\maketitle}{ 
\begin{flushright} % Right align
{\LARGE\@title} % Increase the font size of the title
\vspace{50pt} % Some vertical space between the title and author name
{\large\@author} % Author name
\\\@date % Date
\vspace{40pt} % Some vertical space between the author block and abstract
\end{flushright}
}



%----------------------------------------------------------------------------------------
%	TITLE
%----------------------------------------------------------------------------------------

\title{\textbf{GPS navigation devices }\\ % Title
and Postman’s The Judgement of Thamus } % Subtitle

\author{\textsc{Caroline D. Hardin} % Author
\\{\textit{University of Wisconsin-Madison}}} % Institution

\date{\today} % Date


%----------------------------------------------------------------------------------------

\begin{document}

\maketitle % Print the title section

%----------------------------------------------------------------------------------------
%	ESSAY BODY
%----------------------------------------------------------------------------------------
\cite[para.~15]{Capps2014}

Once, only US Submarines could use satellites for navigation, but now over 65\% of American households can use it with their GPS enabled smartphones (Neisen, 2014, p. 5). While it once took hours to connect, now it’s instantaneous. This 50 year process has transformed the way we navigate from place to place (wayfinding) in many profound and still evolving ways. As a society we struggle today with understanding and balancing many tradeoffs of this new technology, including the question: Is quickly achieving the easiest route worth the loss of skills, exploration and adventure that helps us deeply know our spaces?
	
The benefits of GPS enabled cell phones range from the obvious (reducing stress and saving time and fuel) to the imaginative (augmented reality games, Ushahidi for crowdsourcing social activist mapping, or geo-tagging photos after a disaster or emergency (Konkel, 2014)). However, these gee-wiz features can distract us from thinking about the ecological effect of the more mundane aspects of GPS: always knowing where you are, and how to get somewhere else. After 50 years of increasing saturation, we can see unintended consequences of using GPS to navigate everywhere we go.

In the first chapter of Neil Postman’s of his book Technopoly, The Judgment of Thamus, he explores Plato’s essay about King Thamus judging the inventions of the Egyptian God Theuth. When it came to the invention of writing, King Thamus warned that: ”Those who acquire it will...rely on writing to bring things to their remembrance by external signs instead of by their own internal resources...And as for wisdom, your pupils will have the reputation for it without the reality….”. (Postman, 1992, p. 12). Wayfinding was not immune from transformation with the advent of literacy, and as Thamus predicted, the price of convenience was a decline in orienteering and navigation skills and the fostering of false wisdom in that people could, with a map, feel like they knew a place they had never even visited. Still, maps were limited: they generally required up-front planning and decoding, contained limited information, and showed a snapshot fixed in time whose usefulness decayed, sometimes quite rapidly. 

Now we have invented a technology which transcends the written map - phones with GPS powered navigation apps. GPS navigation apps are fundamentally different from paper maps in several ways: they show you where you are, can calculate directions for getting somewhere else, offer the possibility of accessing any digitally-available information about places, and all with real-time updates. We can see how Thamus’s warning about writing is also applicable to this new technology - not just through anecdotes and essays - but with scientific research into the way our brains perform wayfinding.

The effect of consumer-available technology on wayfinding skills has been studied since the very first systems were made available - a 1999 paper correctly predicted that “...it is important to note that, while using such tools may streamline the wayfinding process, they may also limit the amount of spatial knowledge acquired from the environment.” (Chen \& Stanney, 1999, pg. 682). Skills in orienteering, map reading, and spatial memory have all declined in the age of the GPS navigation system (Aporta \& Higgs, 2005)(Gardony, Brunye, Mahoney, \& Taylor, 2013).  These effects may not be surprising, but Postman reminds us the effects of a new technology can have implications far beyond what is originally apparent. Some may even say it’s not terribly tragic to lose a skill such as map reading when the base technology of maps appears to be facing obsolescence anyway. However, the decline of spatial memory from the use of navigational aids turns out to be important for more than just map reading. Research shows the “...fundamental importance of spatial thinking” (Committee on the Support of Learning Spatially, 2006, p. 13) to a variety of non-navigation applications (Center for Talented Youth at John Hopkins University, n.d., p. 1) such as “autobiographical memories and to imagine the future” (Hutchinson, 2009, para. 26). 

Thamus didn’t just warn about loss of skills, he also warned about the loss of wisdom. When it comes to our geographic spaces, many have lamented in anecdote and research about the decline in deep understanding of local geography. Postman writes that “...embedded in every tool is an ideological bias...” (Postman, 1992, p. 13), and it could be claimed that the ideological bias in GPS navigation is that the destination is more important than the journey.

Even with static maps, as early as 1931 Harry Beck realized when he re-invented the London subway map that, “...passengers riding the Underground were not too bothered about geographical accuracy, and were more interested in how to get from one station to another and where to change trains.” (London Transport Museum Shop, n.d.). How much more so with turn-by-turn navigation! For all the known benefits of faster and easier navigation, “...there is a risk of turning landscapes into constructed entities or commodities, which is what happens figuratively when we are too attentive to the map and not the territory.”  (Aporta \& Higgs, 2005, p. 729).  Not having to learn about your path is not only a convenience, it is also a loss of opportunity to explore and deeply experience the landscape you are passing through. One writer reflects that, “...as I tried frantically to remember the GPS’s instructions, I realized that despite multiple trips to and from work, I had learned exactly nothing about the city’s geography…” (Neyfakh, 2013, para. 3). Another wrote, “Isn’t it ironic: the easier it is for me to get where I’m going, the less I remember how I got there.” (Grabar, 2014, para. 12).  Research supports these observations: a GPS navigated driver is one disengaged with the landscape they are traveling through (Leshed,  Velden, Rieger, Kot, \& Sengers, 2008). 

The effects of using GPS navigation is not just practical, but cultural. John Huth, author of “The Lost Art of Finding Our Way”, describes the decline of deeply knowing the spaces we travel through as: “The loss is aesthetic as much as anything else.” (Neyfakh, 2013, para. 19). Consider the Black Cab drivers of London, who meticulously memorize all 25,000 streets in the six-mile radius from Charing Cross, London and learn how to get from any one street to another. In timed contests, GPS is amazingly, for now, the loser against those who have achieved this Knowledge of London. Yet, the service the cab drivers provide is deeper than just street navigation - they also learn all locations on those streets a passenger might wish to be taken and thus can advise passengers on the whole of London cultural spaces with a white-glove panache no app can rival (Rosen, 2014).

Additionally, how do we quantify how these drivers (and the thousands of aspirants) contribute to the cultural fabric of the city with their years of dedicated study? Surely it helps tie the people and their geography more tightly together, not just with the practical accomplishment of learning where each famous restaurant and historic statue is, but by representing to the population as a whole that such knowledge is worth knowing. “The Knowledge should be maintained because it is good for London’s soul, and for the souls of Londoners….London’s taxi driver test enshrines knowledge as -- to use the au courant term -- an artisanal commodity, a thing that’s local and homespun, thriving ideally in the individual hippocampus, not the digital hivemind” (Rosen, 2014, para. 169). 

Beyond navigation skills and wisdom of local geography, once one begins to think about the larger dystopian possibilities of everyone carrying a GPS device with them we see how true it is that “One can see the failure to critically examine social and cultural, and I would add mental and physical, implications of GPS navigation…” (Robbins, 2013, p. 92). Existing issues include government tracking (Gellman \& Sitani, 2013), malicious or unknown agent tracking (O’Connor, 2013), culturally elitist ‘avoid bad neighborhood’ apps (Capps, 2014), people almost dying from using poor GPS directions (Thompson, 2012), sensitive cultural sites are being disturbed by GPS-enabled tourists (Potterfield, 2006), to name just a few. The neurologist Véronique Bohbo even warns that, “In the next twenty years, I think we’re going to see dementia occurring earlier and earlier.” (Hutchinson, 2009, para. 25).  With these frightening possibilities and all the issues explored above, should we resist the seduction of our GPS systems? 

Postman advises us that, “...it is a mistake to suppose that any technological innovation has a one-sided effect. Every technology is both a burden and a blessing; not either-or, but this-and-that.” (Postman,1992, p. 5). Part of these issues can be resolved with design: a number of the researchers who looked at how GPS navigation deteriorates our spatial skills have pointed out that it would be possible to design a GPS system with a change in emphasis from making navigation as mindless as possible to using the system to gradually train geography. Other studies explore the possibility of “...a mapping tool that suggest[s] pleasant routes based not only on aesthetics, but also on memories, smells and sounds” (Quercia, 2014)( Quercia, Schifanella, \& Aiello 2014). One designer has prototyped an app that encourages users to explore new paths between their destinations (Grabar, 2012).

Tools exist today to increase the positive benefits of GPS navigation such as the augmented reality game Ingress, which requires players to walk to historic or cultural locations giving both exercise and a deeper appreciation of local geography. The Siftr.org website and app developed at the University of Wisconsin Madison features social tagging of interesting local features. University professors are using it to encourage students to engage with their surroundings through the theoretical lens of their coursework.

Other ways to make ubiquitous GPS navigation devices positive for society include laws protecting the privacy of individuals from unwarranted police surveillance, and a requirement for app developers to ramp up their security. Publically owned and developed GPS databases (such as OpenStreetMap project) could protect maps from being manipulated for the agendas of the powerful (as was done in the era of paper maps [Modelski, 1975, para. 14]). We can also as individuals choose to sometimes leave our GPS enabled devices at home, and just explore and wander, remembering that “Getting lost can lead to unexpected adventures” (Williams, 2011, para. 15).

Neil Postman’s warnings in ‘The Judgement of Thamus’ to critically examine our technology, and “...see things whole in their psychic, emotional and moral dimensions” (Postman, 1992, p. 118) is particularly relevant when considering the impact that GPS enabled cell phones have on the ways we navigate through our world. Whether impacts have been variously positive, damaging, or ambiguous, they have been significant, and we would be wise to be thoughtful about them, “For it is inescapable that every culture must negotiate with technology, whether it does so intelligently or not” (Postman, 1992, p. 5).

\bibliography{gpsPostman}
\bibliographystyle{apacite}



\end{document}
